%%% A template to produce a nice-looking Curriculum Vitae.
%%% Kieran Healy <kjhealy@gmail.com>
%%% Most recent version is at http://kjhealy.github.com/kjh-vita
%%%
%%% ------------------------------------------------------------------------
%%% Requirements that are included in a modern tex distribution:
%%% ------------------------------------------------------------------------
%%% xelatex
%%% fontspec.sty
%%% hyperrref.sty
%%% xunicode.sty
%%% color.sty
%%% url.sty
%%% fancyhdr.sty
%%% memoir.cls
%%% fontawesome.sty
%%%
%%%
%%%
%%% ------------------------------------------------------------------------
%%% Requirements from https://github.com/kjhealy/latex-custom-kjh
%%% ------------------------------------------------------------------------
%%% org-preamble-xelatex.sty
%%% memoir-article-styles.sty
%%%
%%% ------------------------------------------------------------------------
%%% Optional
%%% ------------------------------------------------------------------------
%%% git
%%% vc.sty
%%% revnum.sty
%%% Fonts
%%%
%%% ------------------------------------------------------------------------
%%% Note
%%%------------------------------------------------------------------------
%%% Because this is a hand-tweaked file, be on the look out for \medksip,
%%% \bigskip and \newpage commands here and there, which are used to balance
%%% the layout or avoid widows & orphans, etc. You should of course add or
%%% remove these as needed.
%%%------------------------------------------------------------------------

\documentclass[11pt,article,oneside]{memoir}
\usepackage{org-preamble-xelatex}
\usepackage{fontawesome,url}

\usepackage[top=0.75in,bottom=1.5in]{geometry}

%%%------------------------------------------------------------------------
%%% Metadata
%%%------------------------------------------------------------------------

%% Change as needed. Or just add me as a coauthor. Only some of these are
%% used below in the hyperref declaration and address banner section.
\def\myauthor{Michael Steeves}
\def\mytitle{Resume}
\def\mycopyright{\myauthor}
\def\mykeywords{}
\def\mybibliostyle{plain}
\def\mybibliocommand{}
\def\mysubtitle{}
\def\myaffiliation{Some Affiliation}
\def\myaddress{Some Address}
\def\myemail{michael@steeves.io}
\def\myweb{http://steeves.io}
\def\myphone{(508) 728-6592}
\def\myfax{(508) 230-8062}
\def\mytwitter{@steevmi1}
\def\mygit{steevmi1}
\def\myversion{}
\def\myrevision{}


\def\myaffiliation{Some Affiliation}
\def\myauthor{Michael Steeves}
\date{} % not used (revision control instead)
\def\mykeywords{Michael Steeves, Vita, CV, Resume, HPC, Scientific Computing}

%%%------------------------------------------------------------------------
%%% Git version tracking
%%%------------------------------------------------------------------------

%% If you don't use git or the vc package (from CTAN), comment this out.
%% If you comment it out, be sure to remove the \rfoot comment below, too.
\input{vc}

%%%------------------------------------------------------------------------
%%% Document
%%%------------------------------------------------------------------------
\begin{document}

%% Choose fonts for use with xelatex
%% Minion and Myriad are widely available, from Adobe.
%% Pragmata is available to buy at http://www.fsd.it/fonts/pragma.htm
%% and is worth every penny. Any good monospace font will work fine, though.
%% Consolas or inconsolata are good alternatives.
\setromanfont[Mapping={tex-text},
	Numbers={OldStyle},
	Ligatures={Common}]{Minion Pro}
\setsansfont[Mapping=tex-text,
	Ligatures={Common},
	Colour=AA0000]{Myriad Pro}
\setmonofont[Mapping=tex-text,Scale=0.72]{Inconsolata}

\newfontface\scheader[SmallCapsFont={Minion Pro},SmallCapsFeatures={Letters=SmallCaps}]{Minion Pro}

\newfontface\addressblock[Mapping={tex-text},
	Numbers={OldStyle},
	Ligatures={Common}]{Minion Pro}


%%%------------------------------------------------------------------------
%%% Local commands
%%%------------------------------------------------------------------------

%% Marginal header
%% Note: as the document goes on you may need to introduce a (gradually increasing)
%% \vspace element to keep the marginal header pleasingly aligned with the first
%% item in the body text. Like this: \marginhead{{\vskip 0.4em}Grants}, or
%% \marginhead{{\vskip 0.8em}Service}. Experiment as needed.
\newcommand{\marginhead}[1]{\marginpar{\textsf{{\footnotesize\vspace{-1em}\flushright #1}}}}


%% [optional] custom ampersand (font consistent with the one chosen above)
% \newcommand{\amper}{{\fontspec[Scale=.95,Colour=AA0000]{Minion Pro Medium}\selectfont\&\,}}

%% No bullets on labels
\renewcommand{\labelitemi}{~}

%% Custom hanging indent for vita items
\def\ind{\hangindent=1 true cm\hangafter=1 \noindent}
%\def\ind{\hangindent=18pt\hangafter=1 \noindent}
\def\labelitemi{~}
\renewcommand{\labelitemii}{~}

%%%------------------------------------------------------------------------
%%% Page layout
%%%------------------------------------------------------------------------

% These lines will insert git revision info in the footer, using the vc
% package---see docs for vc package for details. Comment out this line
% if you're not using vc, and also remove the \input{vc} line above.
\pagestyle{kjh}
\thispagestyle{kjhgit}


%%%------------------------------------------------------------------------
%%% Address and contact block
%%%------------------------------------------------------------------------
\begin{minipage}[t]{2.95in}
 \flushright {\footnotesize
 4 Pierce Way, \\ \vspace{-0.05in} North Easton, \textsc{ma} 02356.}

\end{minipage}
\hfill
%\begin{minipage}[t]{0.0in}
% dummy (needed here)
%\end{minipage}
\hfill
\begin{minipage}[t]{1.3in}
  \flushright \footnotesize  \addressblock \myphone \, \faPhone \\
%%  {\scriptsize  \texttt{\href{http://twitter.com/steevmi1}{\mytwitter}} \, \faTwitter }  \\
  {\scriptsize  \texttt{\href{mailto:\myemail}{\myemail}} \, \faEnvelope} \\
  {\scriptsize  \texttt{\href{\mygit}{\mygit}} \, \faGithub}
\end{minipage}

\medskip

%% Name
\noindent{\LARGE\scheader \textsc{michael steeves}}
\reversemarginpar

\bigskip


%% Summary
\marginhead{\sffamily summary}

\ind Senior technologist with extensive experience working in the life sciences and pharmaceutical industry helping to define and build consensus around High Performance and Scientific Computing solutions.

\bigskip

%% Education

\marginhead{\sffamily {{\vskip -0.23em} education}}

\ind M.S., Bioinformatics, Brandeis University, 2014.

\ind B.S., Mathematics, University of Massachusetts at Lowell, 1997.

\bigskip

%% Certifications

\marginhead{\sffamily {{\vskip -0.23em} certifications}}
\ind \textbf{Machine Learing}, License \href{https://www.coursera.org/account/accomplishments/verify/Q55XPUZ3LN6S}{Q55XPUZ3LN6S} (Coursera, October 2016)

\ind \textbf{Neural Networks and Deep Learning}, License \href{https://www.coursera.org/account/accomplishments/verify/E92HSAPQB5A2}{E92HSAPQB5A2} (Coursera, February 2018)

\ind \textbf{Improving Deep Neural Networks: Hyperparameter tuning, Regularization and Optimization}, License \href{Improving Deep Neural Networks: Hyperparameter tuning, Regularization and Optimization}{G7CBYGZUHAN5} (Coursera, April 2018)
\bigskip

%% Career
\marginhead{\sffamily {\vskip 0.35em} employment highlights}
\medskip
\noindent\emph{Biogen, Inc. (November, 2021 - present) \vspace{0.01in}}

\ind \footnotesize Director of High Performance Computing and Data Platform (November, 2021 - present)

\ind \hspace{0.35in} \footnotesize Lead a team of 1 Biogen FTE and 9 MSP resources across multiple vendors to support the HPC environment used by Biogen researchers. Lead optimizations to generate 25\% monthly savings on cloud spend. Standardize DevOps approaches to managing the HPC on-prem and cloud systems.

\vspace{-0.075in}

\normalsize

\bigskip
\noindent\emph{The BioTeam, Inc. (March, 2019 - November, 2021) \vspace{0.01in}}

\ind \footnotesize Director, Scientific Computing (November, 2020 - November, 2021)

\ind \hspace{0.35in} \footnotesize Lead for HPC and Scientific Computing practice at BioTeam, handling both hiring and standards and practices in the area of HPC and Scientific Computing. Day-to-day management of consultants from across BioTeam's practice areas. Continued participation in consulting projects for client projects ranging
from assessment recommendations, documentation of solutions and designs, building and testing solutions, and qualifying them for production. Lead for the BioTeam Convergence Lab located at the University of Texas at Austin. Non-voting member of Intel HPC XPharma initiative.

\medskip
\ind \hspace{0.35in} \footnotesize 2021 Member of SuperComputing SCInet Security Team
\medskip

\ind \footnotesize Senior Scientific Consultant (March, 2019 - November, 2020)

\ind \hspace{0.35in} \footnotesize Design, architect, and implement high performance and research computing solutions for client projects ranging from assessment recommendations, documentation of solutions and designs, building and testing solutions, and qualifying them for production. Lead for the BioTeam Convergence Lab located at the University of Texas at Austin. Non-voting member of Intel HPC XPharma initiative.

\vspace{-0.075in}

\normalsize

\bigskip
\noindent\emph{Novartis Institutes for BioMedical Research (July, 2006 - March, 2019) \vspace{0.01in}}

\ind \footnotesize Global Lead, Scientific Computing Group (May, 2018 - March, 2019)

\ind \hspace{0.35in} \footnotesize Manage global team of 13 responsible for HPC strategy, design, integration and operation for informaticians and data scientists. Novartis representative at the Intel HPC XPharma initiative.

\medskip
\ind \footnotesize Architect, Scientific Computing Group (August, 2010 - May, 2018)

%%\ind  Kieran Healy. 2006. \emph{\href{http://www.lastbestgifts.com}{Last Best Gifts: Altruism and the Market for Human Blood and Organs}}. Chicago:~University of Chicago Press. \vspace{0.05in}

\ind \hspace{0.35in} \footnotesize Senior member of 15 person global team providing services for high performance and scientific computing, from infrastructure design through algorithm development. Lead for high performance data analytics and machine learning environment and strategy, HPC in the cloud, and build and support for Informatics applications. Provide direction and third-level support for operational staff. Non-voting member of Intel HPC XPharma initiative.

\medskip
\ind \hspace{0.35in} \footnotesize \textbf{Awards and Presentations}

\ind \hspace{0.35in} \footnotesize 2014 HPC Wire Award for Best Use of Cloud for HPC

\ind \hspace{0.35in} \footnotesize 2013 \href{https://www.youtube.com/watch?v=--ALfYpw_aM}{Speaker, AWS re:Invent BDT212} \emph{Real-world Cloud HPC at Scale, for Production Workloads}

\ind \hspace{0.35in} \footnotesize 2012 Novartis IT Innovation Award for HPC Cloud

\ind \hspace{0.35in} \footnotesize 2011 Novartis IT Innovation Award for Cloud Computing
\medskip

\ind \footnotesize Engineer, Research Computing Platforms (July, 2006 - August, 2010)

\ind \hspace{0.35in} \footnotesize Site support for 1600 users, 400 servers and over 200 TB of storage. Lead for Linux DBA support, monitoring, Cambridge business continuity planning and global user account harmonization. Key member of team that migrated IT infrastructure from on campus to professional data center with minimal downtime and no data loss.

\vspace{-0.075in}

\normalsize

\bigskip
\noindent\emph{Tufts University (Oct, 2004 - July, 2006) \vspace{0.05in}}

\ind \footnotesize Unix Network Administrator

\ind \hspace{0.35in} \footnotesize Member of 4 person team responsible for infrastructure (e-mail, user accounts, web services, 50 servers, over 100 desktops, class/lab computing, storage/backup and network) for ECE and CS departments.

\vspace{-0.075in}

\normalsize

\bigskip
\noindent\emph{Vertex Pharmaceuticals (2001 - Oct, 2004) \vspace{0.05in}}

\ind \footnotesize Senior Unix System Administrator

\ind \hspace{0.35in} \footnotesize Responsible for Unix infrastructure (scientific servers/workstations, FDA-regulated servers, e-mail infrastructure and 50-node Linux parallel compute cluster), SAN and NAS storage.

\vspace{-0.075in}

\normalsize

\bigskip
\noindent\emph{GTE Internetworking/Genuity (July 1999 - 2001)}

\ind \footnotesize Unix System Administrator

\ind \hspace{0.35in} \footnotesize Responsible for a dozen servers and over 150 desktops for several engineering departments, including e-mail and user accounts.

\vspace{-0.075in}

\normalsize

\bigskip
\noindent\emph{Applix, Inc. (1996 - 1999)}

\ind \footnotesize System and Network Engineer (1998 - 1999)

\ind \hspace{0.35in} \footnotesize Responsible for heterogeneous Unix server and workstation environment, LAN/WAN network, e-mail and PBX/Meridian Mail for
corporate headquarters and internal support for networks, phone system, servers and Applixware and remote assistance to European offices.

\ind \footnotesize Customer Support Engineer (1996 - 1998)

%% Use revnumerate environment if numbered publications are needed.
%% (Include it above in the preamble).
%% \renewcommand{\labelenumi}{\textsc{a}\theenumi.}
%% \begin{revnumerate}

%\end{revnumerate}
%\newpage
%%\bigskip

%%\noindent\emph{Book chapters \vspace{0.05in}}
% \renewcommand{\labelenumi}{\textsc{c}\theenumi.}
% \begin{revnumerate}

%\end{revnumerate}

%%\bigskip

%\newpage
%%\noindent\emph{Essays and reviews \vspace{0.05in}}

%\renewcommand{\labelenumi}{\textsc{r}\theenumi.}
%\begin{revnumerate}

% %\end{revnumerate}
%%\bigskip

%% Presentations
%%\marginhead{\sffamily {\vskip 0.5em}selected \newline invited talks \newline since 2007}
%%\medskip


%%\bigskip

%\newpage

%%\marginhead{\sffamily {\vskip 0.5em}selected \newline conference \newline presentations \newline since 2007}
%%\medskip
%%
%%\bigskip

%%\marginhead{\sffamily {\vskip 0.6em}grants,\newline honors, \newline \& awards}
%%\medskip

%%\marginhead{\sffamily {\vskip 1.12em}service to the \newline profession}
%%\medskip

%%\medskip
%%\marginhead{{\vskip 0.9em}Service to the \newline Profession}
%%\medskip

%%\medskip

\end{document}
